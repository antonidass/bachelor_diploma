\chapter{Технологический раздел}
В данном разделе приводится постановка задачи, описываются требования к разрабатываемому методу
и программному комплексу. Рассматривается архитектура нейронной сети и структура программного обеспечения. Описываются данные для обучения модели.

\section{Постановка задачи}

На рисунке \ref*{img:1} приведена постановка задачи в виде IDEF0-диаграммы.
\imgs{1}{H}{1}{Постановка задачи в виде IDEF0–диаграммы}{}{}


\section{Требования к разрабатываемому методу}
Метод распознавания сетевых атак должен:

\begin{itemize}
    \item принимать на вход данные о сетевых параметрах в формате CSV;
    \item определять наличие сетевой атаки во входных данных;
    \item определять вероятность принадлежности атаки к определенному классу.
\end{itemize}

\section{Требования к разрабатываемому программному комплексу}
Программный комплекс, реализующий интерфейс для разработанного
метода, должен предоставлять:

\begin{itemize}
    \item возможность загрузки параметров сетевого трафика через графический интерфейс;
    \item возможность просмотра информации о сетевом трафике;
    \item возможность просмотра информации о количестве атак и их типах в исходных данных.
\end{itemize}


\section{Набор данных CICIDS2017}
Набор данных CICIDS2017 содержит доброкачественные и самые
современные распространенные атаки.
Он также включает в себя результаты анализа сетевого трафика с маркированными потоками на основе
временной метки, портов источника и
назначения, протоколов и атак. Описание параметров приведено в таблице \ref{tbl:inputdata}.

% Настройка выравнивание колонки по центру
\newcolumntype{P}[1]{>{\centering\arraybackslash}p{#1}}

\begin{center}
    \captionsetup{justification=raggedleft,singlelinecheck=off}
    \begin{longtable}[c]{|P{1cm}|P{5cm}|P{10cm}|}
    \caption{Описание параметров датасета CICIDS2017\label{tbl:inputdata}}
    \\ \hline
        \textbf{№} & 
        \textbf{Название параметра} &
        \textbf{Описание} 
    \\ \hline
        1 &
        Flow Duration &
        Продолжительность потока
    % \\ \hline
    %     2 &
    %     Source IP &
    %     IP источника
    % \\ \hline
    %     3 &
    %     Destination IP &
    %     IP назначения
    \\ \hline
        2 &
        Source Port &
        Порт источника
    \\ \hline
        3 &
        Destination Port &
        Порт назначения
    \\ \hline
        4 &
        Protocol ID &
        ID протокола
    % \\ \hline
    %     7 &
    %     Flow Start Milliseconds &
    %     Время начала потока, мс
    \\ \hline
        5 &
        Total Fwd Packets &
        Всего пакетов в прямом направлении
    \\ \hline
        6 &
        Total Backward Packets &
        Всего пакетов в обратном направлении
    \\ \hline
        7 &
        Fwd Packet Length Min &
        Минимальный размер пакета в прямом направлении
    \\ \hline
        8 &
        Fwd Packet Length Max &
        Максимальный размер пакета в прямом направлении
    \\ \hline
        9 &
        Fwd Packet Length Mean &
        Средний размер пакета в прямом направлении
    \\ \hline
        10 &
        Fwd Packet Length Std &
        Размер стандартного отклонения пакета в прямом направлении
    \\ \hline
        11 &
        Fwd IAT Total &
        Общее время между двумя пакетами, отправленными в прямом направлении
    \\ \hline
        12 &
        Fwd IAT Mean &
        Среднее время между двумя пакетами, отправленными в прямом направлении
    \\ \hline
        13 &
        Fwd IAT Std &
        Время стандартного отклонения между двумя пакетами, отправленными в прямом направлении
    \\ \hline
        14 &
        Fwd IAT Max &
        Максимальное время между двумя пакетами, отправленными в прямом направлении
    \\ \hline
        15 &
        Fwd IAT Min &
        Минимальное время между двумя пакетами, отправленными в прямом направлении
    \\ \hline
        16 &
        Fwd IAT Mean &
        Среднее время между двумя пакетами, отправленными в прямом направлении
    \\ \hline
        17 &
        SYN Flag Count &
        Количество пакетов с SYN
    \\ \hline
        18 &
        ACK Flag Count  &
        Количество пакетов с ACK
    \\ \hline
        19 &
        FIN Flag Count &
        Количество пакетов с FIN
    \\ \hline
        20 &
        CWE Flag Count &
        Количество пакетов с CWE
    \\ \hline
        21 &
        Bwd Packet Length Min &
        Минимальный размер пакета в обратном направлении
    \\ \hline
        22 &
        Bwd Packet Length Max &
        Максимальный размер пакета в обратном направлении
    \\ \hline
        23 &
        Bwd Packet Length Mean &
        Средний размер пакета в обратном направлении
    \\ \hline
        24 &
        Bwd Packet Length Std &
        Размер стандартного отклонения пакета в обратном направлении
    \\ \hline
        25 &
        Bwd IAT Total &
        Общее время между двумя пакетами, отправленными в обратном направлении
    \\ \hline
        26 &
        Bwd IAT Mean &
        Среднее время между двумя пакетами, отправленными в обратном направлении
    \\ \hline
        27 &
        Bwd IAT Std &
        Время стандартного отклонения между двумя пакетами, отправленными в обратном направлении
    \\ \hline
        28 &
        Bwd IAT Max &
        Максимальное время между двумя пакетами, отправленными в обратном направлении
    \\ \hline
        29 &
        Bwd IAT Min &
        Минимальное время между двумя пакетами, отправленными в обратном направлении
    \\ \hline
        30 &
        Bwd IAT Mean &
        Среднее время между двумя пакетами, отправленными в обратном направлении
    \\ \hline
        31 &
        Attack Label &
        Метка атаки
    \\ \hline
\end{longtable}
\end{center}


\section{Структура разрабатываемого программного комплекса}

% На рисунке \ref*{img:idef0a0} представлена диаграмма IDEF0 метода обнаружения.
% \imgs{idef0a0}{H}{1}{IDEF0–диаграмма метода обнаружения}{}{}

На рисунке \ref*{img:idef0training} представлена диаграмма IDEF0 обучения модели.
\imgs{idef0training}{H}{0.9}{IDEF0–диаграмма обучения модели}{}{}


\section{Структура многослойной нейронной сети}

Поскольку некоторые семейства атак в сети проще обнаружить, 
чем другие, архитектура многослойной нейронной сети будет модифицирована так, 
чтобы для легко классифицируемых образцов сеть не должна была оценивать все слои.

Такая архитектура основана на
предположении о том, что нейронные сети с большим количеством слоев могут
обучаться все более сложным функциям, которые, в свою очередь, 
требуются только для классификации некоторых конкретных, заведомо
трудноклассифицируемых образцов.

В результате сеть строится таким
образом, что к каждому слою подключается дополнительный набор нейронов
(копия выходных нейронов), позволяющий осуществлять предсказания на каждом уровне.

Данная архитектура представлена на рисунке \ref*{img:eager}

\imgs{eager}{H}{0.9}{Модификация многослойной нейронной сети}{}{}

Исходные нейроны показаны серым цветом, а выходные нейроны (для каждого слоя) --- зеленым. 
Нейронная сеть продолжает оценивать слои один за другим и выдает значение достоверности прогноза на каждом слое.
Значение достоверности определяется как величина, которая отражает, насколько
сеть уверена в принадлежности образца  к определенному классу. Это число от 0,5 до 1, которое
получается путем применения сигмовидной функции к выходному нейрону.

Оценивая значение достоверности, нейронная сеть определяет, следует ли обрабатывать дополнительные 
слои или полученное в данный момент значение достоверности достаточно высоко. Эта модификация гарантирует, 
что более простые образцы будут классифицироваться на ранних слоях, 
а более сложные образцы будут передаваться в более глубокие слои.


\section{Средства реализации программного комплекса}

\subsection{Выбор языка программирования}
Для написания программного обеспечения будет использоваться язык программирования Python \cite[]{python} версии 3.11.

Данный выбор обусловлен следующими факторами:
\begin{itemize}
    \item наличие большого количества библиотек для работы с нейронными сетями;
    \item кроссплатформенность;
    \item автоматическое управление памятью.
\end{itemize}

\subsection{Используемые библиотеки}
При разработке программного обеспечения использовались следующие
библиотеки:
\begin{itemize}
    \item PyTorch \cite{torch} --- библиотека, для создания и обучения модели нейронной сети. Основным преимуществом PyTorch является гибкость. PyTorch предоставляет множество инструментов для разработки своих моделей, включая выбор оптимизаторов, функций потерь и слоев;
    \item PyQt5 \cite{pyqt} --- библиотека для создания графического интерфейса;
    \item Scikit-learn \cite{scikit}  --- библиотека включает в себя подготовку данных для последующей классификации, а также различные алгоритмы машинного обучения и поддерживает взаимодействие с NumPy;
    \item Numpy \cite{numpy} --- это библиотека, которая предоставляет функциональность для работы с многомерными массивами и матрицами. Она используется для научных вычислений, обработки данных и машинного обучения;
    \item Matplotlib \cite{matplot} --- библиотека для визуализация результатов экспериментов и исследований, создание диаграмм и др.
\end{itemize}

\section{Минимальные требования к вычислительной системе}
Для того, чтобы воспользоваться программным обеспечением, потребуется
ЭВМ с предустановленным интерпретатором Python версии 3.11. Также необходимо
скачать и установить все использованные в проекте библиотеки.

Требования к использованию определенной операционной системы отсутствуют поскольку  ПО реализовано на языке Python,
который является кроссплатформенным.


\section{Формат входных данных}
Входными данными являются текстовые файл в формате CSV.
CSV --- формат текстовых данных в котором каждая строка --- это отдельная строка таблицы, а столбцы
отделены один от другого специальными символами-разделителями --- запятыми.
Файл CSV содержит информацию о сетевых параметрах.
Файлы данного формата могут быть открыты на любой операционной системе в
любом программном обеспечении для работы с текстом.

\section{Формат выходных данных}

Выходные данные модуля классификации можно условно разделить на два
типа:
    \begin{itemize}
        \item прогнозирование класса атаки для каждой строки из входного файла;
        \item визуализация количества атак принадлежащих к различным классов сетевых атак.
    \end{itemize}
Полученные выходные данные выводятся на графический интерфейс
программного обеспечения


\section{Структура разработанного ПО}
Структура разработанного ПО изображена на рисунке \ref*{img:modules}.

\imgs{modules}{H}{1}{Диаграмма структуры ПО}{}{}

Каждый модуль, который изображен на диаграмме, содержит в себе
сгруппированные по функциональному значению соответствующие функции и классы.
Модуль взаимодействия с пользователем отвечает за пользовательский
интерфейс. Модуль управления связывает модули классификации, формирования данных, визуализации результатов и тестирования
а также отвечает за координацию их работы.


% \section{Установка программного обеспечения}
% Для запуска разработанного программного продукта требуется установить
% на ЭВМ интерпретатор для Python 3.11 и создать виртуальное окружение для
% ПО. В листинге \ref*{code:text} приведены команды, которые необходимо выполнить для
% создания виртуального окружения.

% \begin{lstlisting}[label=code:text, language=bash, caption={Команды для установки виртуального окружения}]
%     python3 -m pip install -user virtualenv
%     python3 -m venv env
%     source env/bin/activate
% \end{lstlisting}

% Используемые в разработке библиотеки, которые необходимы для запуска
% ПО, приведены в файле \textbf{requirements.txt}, который находится в корневой директории 
% проекта. С помощью пакетного менеджера pip все зависимости нужно
% установить, запустив в терминале команду, приведенную в
% листинге \ref*{code:text2}.

% \begin{lstlisting}[label=code:text2, language=bash, caption={Команда для установки необходимых библиотек}]
%     pip3 install -r requirements.txt
% \end{lstlisting}


\section{Пользовательский интерфейс}
Графический интерфейс разработан при помощи библотеки PyQt5, предоставляющей набор классов и методов для работы с компонентами интерфейса.
На рисунке \ref*{img:gui} представлен пользовательский интерфейс.

\imgs{gui}{H}{0.45}{Пользовательский интерфейс}{}{}
В колонке <<Предполагаемая атака>> отображаются результаты обнаружения сетевых атак во входном CSV-файле.
В колонке <<Вероятность>> отображается вероятнсть принадлежности атаки к соответствующему классу.

Если пользователь, не выбрав входной файл, нажмет на кнопку <<Обнаружить сетевые атаки>> или на кнопку <<Построить гистограмму>>,
то на экран выведется сообщение об ошибке, как показано на рисунке \ref*{img:er}.
\imgs{er}{H}{0.46}{Сообщение об ошибке}{}{}

При нажатии на кнопку «Выбрать файл» открываются файловая система
устройства, после чего необходимо выбрать интересующий CSV-файл.
После выбора файла, сетевые параметры отобразятся в таблице <<Исходные данные>>.

Если файл выбран, то при нажатии на кнопку <<Обнаружить сетевые атаки>>, запустится алгоритм обнаружения сетевых атак для каждой строки из входного CSV-файла. 
Результат работы алгоритма представлен на рисунке \ref*{img:res}.

\imgs{res}{H}{0.46}{Результат работы}{}{}


При нажатии на кнопку <<Построить гистограмму>> на экран выведется гистограмма,
отображающая количество атак принадлежащих к различным классам сетевых атак.
\imgs{hist2}{H}{0.8}{Распределение сетевых атак}{}{}

В листинге \ref*{code:text3} приведены параметры для взаимодействия с модулем обучения модели.

\begin{lstlisting}[label=code:text3, language=bash, caption={Взаимодействие с модулем модели}]
python3 mlearn.py -h

options:
    -h, --help                show this help message and exit
    --dataroot DATAROOT       путь к датасету
    --batchSize BATCHSIZE     размер выборки данных на каждой итерации
    --nLayers NLAYERS         количество слоев
    --layerSize LAYERSIZE     размер слоя
    --niter NITER             количество эпох для обучения
    --net NET                 путь к сохраненной модели
    --manualSeed MANUALSEED   seed для перемешивания данных
    --stoppingWeightingMethod EAGERSTOPPINGWEIGHTINGMETHOD
                              функция весов
\end{lstlisting}

% \section{Тестирование программного обеспечения}
% Проведено тестирование программного обеспечения. Тесты приведены в таблице \ref*{tbl:tests}

% \begin{center}
%     \captionsetup{justification=raggedleft,singlelinecheck=off}
%     \begin{longtable}[c]{|P{5cm}|P{6cm}|P{5.5cm}|}
%     \caption{Сравнение методов \label{tbl:tests}}
%     \\ \hline
%         \textbf{Название функции} &
%         \textbf{Входные данные} &
%         \textbf{Ожидаемый результат} 
%     \\ \hline
%         filter\_input\_data &
%         CSV-файл с информацией о сетевых параметрах &
%         Датасет типа pd.DataFrame \cite{pddataframe} с отфильтрованными сетевыми параметрами
%     \\ \hline
%         predict\_all\_normal &
%         pd.DataFrame с данными о нормальном сетвом поведении  &
%         Массив данных, все элементы которого принадлежат классу Normal
%     \\ \hline
%         predict\_all\_ddos\_loit &
%         pd.DataFrame с данными о сетевых атаках типа DDoS LOIT &
%         Массив данных, все элементы которого принадлежат классу DDoS LOIT
%     \\ \hline
%         predict\_mix\_attacks &
%         pd.DataFrame с данными о всех типах сетевых атаках &
%         Массив данных, элементы которого принадлежат соответствующим классам из входного датасета 
%     \\ \hline
%     predict\_all\_portscan &
%     pd.DataFrame с данными о сетевых атаках типа PortScan &
%     Массив данных, все элементы которого принадлежат классу PortScan
%     \\ \hline
%     predict\_all\_ddos\_hulk &
%     pd.DataFrame с данными о сетевых атаках типа DDoS Hulk &
%     Массив данных, все элементы которого принадлежат классу DDoS Hulk
%     \\ \hline
% \end{longtable}
% \end{center}

\section{Зависимость точности и потерь модели от количества эпох}

На рисунках \ref*{img:accuracy} и \ref*{img:loss} представлены результаты обучения модели. Точность модели составила 92.1 процента, а потери составили 4 процента. Стоит
отметить, что уже на десятой эпохе точность модели составляет более 88 процентов.

\imgs{accuracy}{H}{0.7}{Точность модели}{}{}
\imgs{loss}{H}{0.9}{Потери модели}{}{}

\section{Зависимость точности модели от класса сетевой атаки и количества слоев}
На рисунках \ref*{img:plot1res} и \ref*{img:plot2res} продемонстрирована зависимось точности модели на каждом слое от класса сетевой атаки.


\imgs{plot1res}{H}{0.5}{5 слоев $\times$ 128 нейронов}{}{}

\imgs{plot2res}{H}{0.65}{12 слоев $\times$ 64 нейрона}{}{}

На основе полученных данных можно сделать следующие выводы:

\begin{itemize}
    \item в среднем точность прогнозов возрастает
    по мере того, как увеличивается количество слоев. Это соответствует действительности, 
    поскольку большее количество слоев обеспечивает большую абстракцию
    входного пространства и, следовательно, лучшее разделение экземпляров. Однако для получения максимальной
    точности для некоторых классов атак требуется всего несколько уровней. Это означает, что примененная модификация архитектуры 
    многослойной нейронной сети позволяет 
    значительно сокращать ресурсы при обучении, сохраняя максимальную точность;
    \item некоторые семейства атак (DDoS Heartbeed, Web-XSS, Web SQL injection, PortScan FW-on) демонстрируют крайне низкую точность. Это происходит по двум причинам:
        \begin{enumerate}
            \item Слишком мало выборок данных классов в обучающем множестве.
            \item Потери, оптимизированные для обнаружения таких классов на определенном уровне
                    перезаписываются потерями, оптимизированными для изучения большинства классов что приводит к стиранию информации об таких классах.
        \end{enumerate}
\end{itemize}

\section{Оценка разработанного программного обеспечения}
У разработанного программного обеспечения были выявлены следующие
достоинства и недостатки.

Достоинства:
\begin{itemize}
    \item высокая точность;
    \item возможность сокращения времени обучения модели за счет уменьшения количества слоев.
\end{itemize}

Недостатки:
\begin{itemize}
    \item низкая точность предсказания следующих атак: DDoS Heartbeed, Web XSS, Web SQL injection, PortScan FW-on.
\end{itemize}

\section*{Вывод}
Было разработано программное обеспечение, демонстрирующее практическую осуществимость спроектированного в
ходе выполнения выпускной квалификационной работы метода обнаружения сетевых атак с использованием многослойной нейронной сети.

% Были описаны средства реализации программного комплекса. 
% Описаны входные и выходные данные. Описаны технологии и методы, использовавшиеся при реализации.
% Определены минимальные требования к вычислительной системе. Приведена инструкция
% для установки ПО. Приведено описание интерфейса пользователя. Проведено модульное тестирование.
% Были представлены требования к разрабатываемому методу обнаружения и программному комплексу.

% Представлены схемы работы с разрабатываемыми программными модулями. 

% Представлена модификация архитектуры многослойной нейронной сети.

% Представлен выбор датасета для обучения модели и описаны входные данные.













\chapter{Исследовательский раздел}
В данном разделе проводится оценка точности классификации разработанного метода. Описывается его применимость для классификации различных типов сетевых атак. Приводятся преимущества и недостатки разработанного метода.

\section{Выборка данных}
Для обучения и тестирования модели на вход подаётся 2317923 записи о сетевых атаках.
Для обучения модели была создана обучающая выборка,
составляющая $\frac{2}{3}$ от общего числа записей. Оставшиеся $\frac{1}{3}$ необходимы
для тестирования обученной модели и получения оценок качества классификции.

Однако количество записей об атаках каждого класса неоднородно, поэтому
точность определения на некоторых классах слишком низкая.

\section{Зависимость точности и потерь модели от количества эпох}

На рисунках \ref*{img:accuracy} и \ref*{img:loss} представлены результаты обучения модели. Точность модели составила 92.1 процента, а потери составили 4 процента. Стоит
отметить, что уже на десятой эпохе точность модели составляет более 88 процентов.

\imgs{accuracy}{H}{0.9}{Точность модели}{}{}
\imgs{loss}{H}{0.9}{Потери модели}{}{}

\section{Зависимость точности модели от класса сетевой атаки и количества слоев}
На рисунках \ref*{img:plot1res} и \ref*{img:plot2res} продемонстрирована зависимось точности модели на каждом слое от класса сетевой атаки.


\imgs{plot1res}{H}{0.5}{5 слоев $\times$ 128 нейронов}{}{}

\imgs{plot2res}{H}{0.65}{12 слоев $\times$ 64 нейрона}{}{}

На основе полученных данных можно сделать следующие выводы:

\begin{itemize}
    \item в среднем точность прогнозов возрастает
    по мере того, как увеличивается количество слоев. Это соответствует действительности, 
    поскольку большее количество слоев обеспечивает большую абстракцию
    входного пространства и, следовательно, лучшее разделение экземпляров. Однако для получения максимальной
    точности для некоторых классов атак требуется всего несколько уровней. Это означает, что примененная модификация архитектуры 
    многослойной нейронной сети позволяет 
    значительно сокращать ресурсы при обучении, сохраняя максимальную точность;
    \item некоторые семейства атак (DDoS Heartbeed, Web-XSS, Web SQL injection, PortScan FW-on) демонстрируют крайне низкую точность. Это происходит по двум причинам:
        \begin{enumerate}
            \item Слишком мало выборок данных классов в обучающем множестве.
            \item Потери, оптимизированные для обнаружения таких классов на определенном уровне
                    перезаписываются потерями, оптимизированными для изучения большинства классов что приводит к стиранию информации об таких классах.
        \end{enumerate}
\end{itemize}


\section{Исследование требуемого количества слоев для модели нейронной сети}

Во время обучения модели пользователь может установить оптимальное значение достоверности. 
Установка слишком низкого значения может привести к тому, что нейронная сеть будет принимать решения 
на ранней стадии, тем самым используя меньше ресурсов, но при этом точность модели будет низкой.
Аналогично, установка слишком высокого значения может привести к тому, что сеть
всегда будет использовать все уровни, что приведет к пустой трате вычислительных ресурсов при обучении.
Таким образом, желаемый уровень достоверности может быть выбран из результатов эксперимента 
показанных на рисунках \ref*{img:plot3res} и \ref*{img:plot4res}.

\imgs{plot3res}{H}{0.85}{12 слоев $\times$ 64 нейрона}{}{}

\imgs{plot4res}{H}{0.55}{3 слоя $\times$ 128 нейронов}{}{}



\section{Оценка разработанного программного обеспечения}
У разработанного программного обеспечения были выявлены следующие
достоинства и недостатки.

Достоинства:
\begin{itemize}
    \item высокая точность;
    \item возможность сокращения времени обучения модели за счет уменьшения количества слоев.
\end{itemize}

Недостатки:
\begin{itemize}
    \item низкая точность предсказания следующих атак: DDoS Heartbeed, Web XSS, Web SQL injection, PortScan FW-on.
\end{itemize}

\section*{Вывод}
Проанализирована зависимость точнсоти обучения модели от количества эпох. Определено требуемое количествах слоев для обнаружения атак принадлежащих к различным классам.  Приведены достоинства и недостатки разработанного программного обеспечения.



\maketableofcontents

\intro


% Быстрый рост информационных технологий вызывает ряд проблем, связанных с
% защитой сетевых ресурсов в глобальной сети. Согласно исследованиям,
% с начала 2022 года количество сетевых атак увеличилось на 15\% по сравнению
% с 2021 годом, а доля массовых атак составляет 33\% от всего числа.
% Также вырос интерес к интернет-ресурсам: доля атак на них увеличилась до 22\% от общего числа в сравнении с наблюдаемыми 13\% в 2021 году \cite{investigate}.

% На основании данного исследования можно сделать вывод, что количество сетевых атак лишь растет, а следовательно, растет и потребность в защите от них.

Во время выполнения выпускной квалификационной работы был разработан метод обнаружения сетевых атак с использованием многослойной нейронной сети.


Цель данной работы --- демонстрация практической осуществимости спроектированного в
ходе выполнения выпускной квалификационной работы метода обнаружения сетевых атак с использованием многослойной нейронной сети.

Для достижения поставленной цели необходимо решить следующие задачи:

\begin{itemize}
    \item описать постановку задачи;
    \item привести ограничения на входные и выходные данные;
    \item описать технологии, с помощью которых был реализован метод обнаружения сетевых атак;
	\item провести анализ программного комплекса, реализующего интерфейс для взаимодействия с разработанным методом;
	\item исследовать разработанный метод на применимость.
\end{itemize}


\chapter{Конструкторский раздел}

В данном разделе описываются требования к разрабатываемому методу
и программному комплексу. Рассматривается архитектура нейронной сети и структура программного обеспечения. Описываются данные для обучения модели.

\section{Требования к разрабатываемому методу}
Метод распознавания сетевых атак должен:

\begin{itemize}
    \item принимать на вход данные о сетевых параметрах в формате CSV;
    \item определять наличие сетевой атаки во входных данных;
    \item определять вероятность принадлежности атаки к определенному классу.
\end{itemize}

\section{Требования к разрабатываемому программному комплексу}
Программный комплекс, реализующий интерфейс для разработанного
метода, должен предоставлять:

\begin{itemize}
    \item возможность загрузки параметров сетевого трафика через графический интерфейс;
    \item возможность просмотра информации о сетевом трафике;
    \item возможность просмотра информации о количестве атак и их типах в исходных данных.
\end{itemize}


\section{Набор данных CICIDS2017}
Набор данных CICIDS2017 содержит доброкачественные и самые
современные распространенные атаки.
Он также включает в себя результаты анализа сетевого трафика с маркированными потоками на основе
временной метки, портов источника и
назначения, протоколов и атак. Описание параметров приведено в таблице \ref{tbl:inputdata}.


\begin{center}
    \captionsetup{justification=raggedleft,singlelinecheck=off}
    \begin{longtable}[c]{|P{1cm}|P{5cm}|P{10cm}|}
    \caption{Описание параметров датасета CICIDS2017\label{tbl:inputdata}}
    \\ \hline
        \textbf{№} & 
        \textbf{Название параметра} &
        \textbf{Описание} 
    \\ \hline
        1 &
        Flow Duration &
        Продолжительность потока
    % \\ \hline
    %     2 &
    %     Source IP &
    %     IP источника
    % \\ \hline
    %     3 &
    %     Destination IP &
    %     IP назначения
    \\ \hline
        2 &
        Source Port &
        Порт источника
    \\ \hline
        3 &
        Destination Port &
        Порт назначения
    \\ \hline
        4 &
        Protocol ID &
        ID протокола
    % \\ \hline
    %     7 &
    %     Flow Start Milliseconds &
    %     Время начала потока, мс
    \\ \hline
        5 &
        Total Fwd Packets &
        Всего пакетов в прямом направлении
    \\ \hline
        6 &
        Total Backward Packets &
        Всего пакетов в обратном направлении
    \\ \hline
        7 &
        Fwd Packet Length Min &
        Минимальный размер пакета в прямом направлении
    \\ \hline
        8 &
        Fwd Packet Length Max &
        Максимальный размер пакета в прямом направлении
    \\ \hline
        9 &
        Fwd Packet Length Mean &
        Средний размер пакета в прямом направлении
    \\ \hline
        10 &
        Fwd Packet Length Std &
        Размер стандартного отклонения пакета в прямом направлении
    \\ \hline
        11 &
        Fwd IAT Total &
        Общее время между двумя пакетами, отправленными в прямом направлении
    \\ \hline
        12 &
        Fwd IAT Mean &
        Среднее время между двумя пакетами, отправленными в прямом направлении
    \\ \hline
        13 &
        Fwd IAT Std &
        Время стандартного отклонения между двумя пакетами, отправленными в прямом направлении
    \\ \hline
        14 &
        Fwd IAT Max &
        Максимальное время между двумя пакетами, отправленными в прямом направлении
    \\ \hline
        15 &
        Fwd IAT Min &
        Минимальное время между двумя пакетами, отправленными в прямом направлении
    \\ \hline
        16 &
        Fwd IAT Mean &
        Среднее время между двумя пакетами, отправленными в прямом направлении
    \\ \hline
        17 &
        SYN Flag Count &
        Количество пакетов с SYN
    \\ \hline
        18 &
        ACK Flag Count  &
        Количество пакетов с ACK
    \\ \hline
        19 &
        FIN Flag Count &
        Количество пакетов с FIN
    \\ \hline
        20 &
        CWE Flag Count &
        Количество пакетов с CWE
    \\ \hline
        21 &
        Bwd Packet Length Min &
        Минимальный размер пакета в обратном направлении
    \\ \hline
        22 &
        Bwd Packet Length Max &
        Максимальный размер пакета в обратном направлении
    \\ \hline
        23 &
        Bwd Packet Length Mean &
        Средний размер пакета в обратном направлении
    \\ \hline
        24 &
        Bwd Packet Length Std &
        Размер стандартного отклонения пакета в обратном направлении
    \\ \hline
        25 &
        Bwd IAT Total &
        Общее время между двумя пакетами, отправленными в обратном направлении
    \\ \hline
        26 &
        Bwd IAT Mean &
        Среднее время между двумя пакетами, отправленными в обратном направлении
    \\ \hline
        27 &
        Bwd IAT Std &
        Время стандартного отклонения между двумя пакетами, отправленными в обратном направлении
    \\ \hline
        28 &
        Bwd IAT Max &
        Максимальное время между двумя пакетами, отправленными в обратном направлении
    \\ \hline
        29 &
        Bwd IAT Min &
        Минимальное время между двумя пакетами, отправленными в обратном направлении
    \\ \hline
        30 &
        Bwd IAT Mean &
        Среднее время между двумя пакетами, отправленными в обратном направлении
    \\ \hline
        31 &
        Attack Label &
        Метка атаки
    \\ \hline
\end{longtable}
\end{center}


\section{Структура разрабатываемого программного комплекса}



На рисунке \ref*{img:idef0training} представлена диаграмма IDEF0 обучения модели.
\imgs{idef0training}{H}{0.9}{IDEF0–диаграмма обучения модели}{}{}


\section{Структура многослойной нейронной сети}

Поскольку некоторые семейства атак в сети проще обнаружить, 
чем другие, архитектура многослойной нейронной сети будет модифицирована так, 
чтобы для легко классифицируемых образцов сеть не должна была оценивать все слои.

Такая архитектура основана на
предположении о том, что нейронные сети с большим количеством слоев могут
обучаться все более сложным функциям, которые, в свою очередь, 
требуются только для классификации некоторых конкретных, заведомо
трудноклассифицируемых образцов.

В результате сеть строится таким
образом, что к каждому слою подключается дополнительный набор нейронов
(копия выходных нейронов), позволяющий осуществлять предсказания на каждом уровне.

Данная архитектура представлена на рисунке \ref*{img:eager}

\imgs{eager}{H}{0.9}{Модификация многослойной нейронной сети}{}{}

Исходные нейроны показаны серым цветом, а выходные нейроны (для каждого слоя) --- зеленым. 
Нейронная сеть продолжает оценивать слои один за другим и выдает значение достоверности прогноза на каждом слое.
Значение достоверности определяется как величина, которая отражает, насколько
сеть уверена в принадлежности образца  к определенному классу. Это число от 0,5 до 1, которое
получается путем применения сигмовидной функции к выходному нейрону.

Оценивая значение достоверности, нейронная сеть определяет, следует ли обрабатывать дополнительные 
слои или полученное в данный момент значение достоверности достаточно высоко. Эта модификация гарантирует, 
что более простые образцы будут классифицироваться на ранних слоях, 
а более сложные образцы будут передаваться в более глубокие слои.

\section{Вывод}
Были представлены требования к разрабатываемому методу обнаружения и программному комплексу.

Представлены схемы работы с разрабатываемыми программными модулями. 

Представлена модификация архитектуры многослойной нейронной сети.

Представлен выбор датасета для обучения модели и описаны входные данные.












